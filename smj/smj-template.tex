%%%
%%%  LaTeX template for publications
%%%  to be submitted to Statistical Modelling
%%%
%%%  Prepared by Arnost Komarek
%%%  Version 0.2 (20140214)
%%%    0.2:  style of references slightly changed,
%%%          support for use with bibTeX added
\documentclass[submit]{smj}


%%%%% PREAMBLE
%%%%% =============================================================================


%%% Place for putting personal \usepackage and \newcommand commands
%%% Note that some packages are loaded automatically
%%% with the smj class. 
%%% These include: graphicx, color, fancyvrb, amsmath, amssymb, calc, upquote (if available), natbib, url, hyperref.
%%%
%%% Please, specify all your personal definitions, newcommand etc. here
%%% and not inside the main body of the text.
%%% -------------------------------------------------------------------------------
%\usepackage{PACKAGE}
%\newcommand{MYCOMMAND}{...}


%%% Identification of authors
%%% -------------------------------------------------------------------------------
%%% For each author, provide his/her first name, surname and possibly initials 
%%% of the middle names. 
%%%
%%% Use \Affil{NUMBER} following the author name for each unique affiliation,
%%% where NUMBER is integer starting from 1 to the number of affiliations needed
%%% in this paper. In case of multiple affiliations of one author, use
%%% \Affil{NUMBER1,}\Affil{NUMBER2,}\Affil{NUMBER3} following the author's name
%%% as it is done for Emmanuel Lesaffre below.

  %%% For papers with 3 or more authors:
  %%%  in \Author{}, separate the authors with commas, the last author is separated by `and' without a comma,
  %%%  in \AuthorRunning{}, use the full name of the first author followed by \textrm{et al.}.
\Author{Kathryn Haglich\Affil{1}, 
        Sarah Neitzel\Affil{2}, 
        and Amy Pitts\Affil{3}
        %and Emmanuel Lesaffre\Affil{4,}\Affil{5}
}
\AuthorRunning{Arno\v{s}t Kom\'arek \textrm{et al.}}

  %%% For papers with 2 authors:
  %%%  in both \Author{} and \AuthorRunning{},
  %%%  use the full names of both authors separated by 'and' without a comma.
%\Author{Arno\v{s}t Kom\'arek\Affil{1} and Brian Marx\Affil{2}}
%\AuthorRunning{Arno\v{s}t Kom\'arek and Brian Marx}
  
  %%% For papers with 1 author:
  %%%  in both \Author{} and \AuthorRunning{},
  %%%  use the full name the author.
%\Author{Arno\v{s}t Kom\'arek\Affil{1}}
%\AuthorRunning{Arno\v{s}t Kom\'arek}


%%% Affiliations as they should appear on the title page.
%%% -------------------------------------------------------------------------------
%%% Do not provide the full addresses here.
%%% The ordering inside \Affiliations{} should correspond to NUMBERs used 
%%% in \Affil{} commands in \Author{}
\Affiliations{

  %%% 1
\item Department of Mathematics, 
      Lafayette College,
      Easton,
      Pennsylvania, USA

  %%% 2
\item School of Biodiversity Conservation,
      Unity College, 
      Unity,
      Maine, USA

  %%% 3
\item Department of Mathematics,
      Marist College,
      Poughkeepsie
      NY, USA

  %%% 4
%\item Department of Biostatistics,
%      Erasmus University Rotterdam,
%      Rotterdam,
%      the Netherlands

  %%% 5
%\item Interuniversity Institute for Biostatistics and Statistical Bioinformatics,
%      KU Leuven and Universiteit Hasselt,
%      Leuven,
%      Belgium
}   %% end \Affiliations


%%% Postal, e-mail address, phone and fax of the corresponding author (not necessarily the first author).
%%% ------------------------------------------------------------------------------------------------------
%%% Use command \CorrAddress{} to provide a full postal address of the
%%% corresponding author in the form
%%% "Firstname Lastname, Department, University, Street 1, ZIP City, Country" 
%%% Use command \CorrEmail{} to provide an e-mail address of the corresponding author.
%%% Use command \CorrPhone{} to provide a phone number (including the country code!) of the corresponding author.
%%% Use command \CorrFax{} to provide a fax number (including the country code!) of the corresponding author.
\CorrAddress{Arno\v{s}t Kom\'arek, 
             Department of Probability and Mathematical Statistics, 
             Faculty of Mathematics and Physics,
             Charles University in Prague, 
             Sokolovsk\'a 83, 
             CZ--186$\,$75 Praha 8 -- Karl\'{\i}n, 
             Czech Republic}
\CorrEmail{smj-komarek@karlin.mff.cuni.cz}
\CorrPhone{(+420)\;221\;913\;282}
\CorrFax{(+420)\;222\;323\;316}


%%% Title and a short title (to be used as a running header) of the paper
%%% -------------------------------------------------------------------------------
\Title{Template paper for submissions in Statistical Modelling with a~rather long title exceeding two lines}
\TitleRunning{Template paper}


%%% Abstract
%%% -------------------------------------------------------------------------------
\Abstract{
An abstract of up to 200 words should precede the text together with 5 or 6 keywords in alphabetical order 
to describe the content of the paper. Authors should take great care in preparing the abstract and not simply 
lift it from the main text. The abstract should describe the background and contribution of the manuscript 
and give a~clear verbal description of the results and examples, and avoid citations whenever possible. 
Any acknowledgements will be printed at the end of the text.
}


%%% Key words
%%% -------------------------------------------------------------------------------
\Keywords{
keyword a; keyword b; keyword c; keyword d; keyword e
}


%%%%% MAIN BODY 
%%%%% =============================================================================
\begin{document}


%%% Title page
%%% -------------------------------------------------------------------------------
%%% Use command\maketitle to produce the title page.
\maketitle


%%% Main text
%%% ------------------------------------------
\section{Title of the section}
Text of the first section. \citet{Fahrmeiret13} is a~direct reference to a~book with more than two authors.
\citet{Gomezet09} is a~direct reference to a~journal article with more than two authors.
Many papers were published in \emph{Statistical Modelling} \citep[see, e.g.,][]{Kneib13, KomarekLesaffre06, Liet07, Waldmannet13}.
Sometimes we also need to reference a~book chapter \citep{Lesaffreet09}.

\subsection{Title of the subsection}

\subsubsection{Title of the subsubsection}


%%% Acknowledgements (if any)
%%% ------------------------------------------
\section*{Acknowledgements}
We want to thank\ldots


%%% References if bibTeX is used
%%%
%%% Please, do not specify any \bibliographystyle{} command!
%%%
%%% It is already specified in the smj.cls and its
%%% second specification here causes error.
%%% ------------------------------------------------------------
\bibliography{smj-template}


%%% References (if created by hand).
%%% -----------------------------------------------------------------------------------
%\begin{thebibliography}{99}
%\bibitem[Fahrmeir et~al.(2013)]{Fahrmeiret13}
%Fahrmeir, L., Kneib, T., Lang, S., and Marx, B. (2013).
%\textit{Regression: Models, Methods and Applications}.
%Springer-Verlag, New York.
%
%\bibitem[G\'omez et al.(2009)]{Gomezet09}
%G\'omez, G., Luz Calle, M., Oller, R. and Langohr, K. (2009).
%Tutorial on methods for interval-censored data and their implementation in {R}.
%\textit{Statistical Modelling}, \textbf{9}, 259--297.
%
%\bibitem[Kneib(2013)]{Kneib13}
%Kneib, T. (2013). Beyond mean regression. \textit{Statistical Modelling}, \textbf{13}, 275--303.
%
%\bibitem[Kom\'arek and Lesaffre(2006)]{KomarekLesaffre06}
%Kom\'arek, A. and Lesaffre, E. (2006). Bayesian semi-parametric accelerated failure time model for paired doubly-interval-censored data.
%\textit{Statistical Modelling}, \textbf{6}, 3--22.
%
%\bibitem[Lesaffre et~al.(2009)]{Lesaffreet09}
%Lesaffre, E., Kom\'arek, A., and Jara A. (2009)
%The Bayesian approach.
%In Lesaffre, E., Feine, J., Leroux, B., and Declerck, D., eds. 
%\textit{Statistical and Methodological Aspects of Oral  Health Research},
%pages 315--338.
%John Wiley and Sons, Chichester.
%
%\bibitem[Li et~al.(2007)]{Liet07}
%Li, L., Simonoff, J. S., and Tsai, C.-L. (2007)
%Tobit model estimation and sliced inverse regression.
%\textit{Statistical Modelling}, \textbf{7}, 107--123.
%
%\bibitem[Waldmann et~al.(2013)]{Waldmannet13}
%Waldmann, E., Kneib, T., Yue, Y. R., Lang, S., and Flexeder, C. (2013)
%Bayesian semiparametric additive quantile regression.
%\textit{Statistical Modelling}, \textbf{13}, 223--252.
%\end{thebibliography}

\end{document}
